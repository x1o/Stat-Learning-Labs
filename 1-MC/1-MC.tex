%% LyX 2.1.4 created this file.  For more info, see http://www.lyx.org/.
%% Do not edit unless you really know what you are doing.
\documentclass[russian]{scrartcl}
\usepackage[T2A,T1]{fontenc}
\usepackage[utf8]{inputenc}
\usepackage[a4paper]{geometry}
\geometry{verbose,tmargin=2cm,bmargin=2.5cm,lmargin=2cm,rmargin=2cm}
\usepackage{color}
\usepackage{babel}
\usepackage{latexsym}
\usepackage{mathrsfs}
\usepackage{mathtools}
\usepackage{amsmath}
\usepackage{amsthm}
\usepackage{amssymb}
\usepackage{esint}
\usepackage[unicode=true,pdfusetitle,
 bookmarks=true,bookmarksnumbered=false,bookmarksopen=true,bookmarksopenlevel=1,
 breaklinks=false,pdfborder={0 0 1},backref=false,colorlinks=true]
 {hyperref}

\makeatletter

%%%%%%%%%%%%%%%%%%%%%%%%%%%%%% LyX specific LaTeX commands.
\DeclareRobustCommand{\cyrtext}{%
  \fontencoding{T2A}\selectfont\def\encodingdefault{T2A}}
\DeclareRobustCommand{\textcyr}[1]{\leavevmode{\cyrtext #1}}
\AtBeginDocument{\DeclareFontEncoding{T2A}{}{}}


\@ifundefined{date}{}{\date{}}
%%%%%%%%%%%%%%%%%%%%%%%%%%%%%% User specified LaTeX commands.
%\usepackage{nicefrac}
%\usepackage{colortbl}
%\usepackage[noend]{algpseudocode}
\usepackage[all]{xy}
\usepackage{mathrsfs}

%\usepackage[columns=1,itemlayout=singlepar,totoc=true]{idxlayout}

%\@addtoreset{chapter}{part}
\DeclareMathOperator{\Int}{Int}
\DeclareMathOperator{\rk}{rk}

\newcommand{\bigperp}{%
  \mathop{\mathpalette\bigp@rp\relax}%
  \displaylimits
}

\newcommand{\bigp@rp}[2]{%
  \vcenter{
    \m@th\hbox{\scalebox{\ifx#1\displaystyle2.1\else1.5\fi}{$#1\perp$}}
  }%
}

\newcommand{\bignparallel}{%
  \mathop{\mathpalette\bignp@rp\relax}%
  \displaylimits
}

\newcommand{\bignp@rp}[2]{%
  \vcenter{
    \m@th\hbox{\scalebox{\ifx#1\displaystyle2.1\else1.5\fi}{$#1\nparallel$}}
  }%
}

\AtBeginDocument{
  \def\labelitemii{\(\Diamond\)}
  \def\labelitemiii{\(\Box\)}
}

\makeatother

\begin{document}
\global\long\def\N{\mathrm{N}}
\global\long\def\P{\mathsf{P}}
\global\long\def\E{\mathsf{E}}
\global\long\def\D{\mathsf{D}}
\global\long\def\O{\Omega}
\global\long\def\F{\mathcal{F}}
\global\long\def\K{\mathsf{K}}
\global\long\def\A{\mathscr{A}}
\global\long\def\Pcal{\mathcal{P}}
\global\long\def\th{\theta}
\global\long\def\toas{\xrightarrow{{\rm a.s.}}}
\global\long\def\toP{\xrightarrow{\P}}
\global\long\def\tod{\xrightarrow{\mathrm{d}}}
\global\long\def\iid{\mathrm{i.i.d.}}
\global\long\def\T{\mathrm{T}}
\global\long\def\L{\mathcal{L}}
\global\long\def\dd#1#2{\frac{\mathrm{d}#1}{\mathrm{d}#2}}
\global\long\def\a{\alpha}
\global\long\def\b{\beta}
\global\long\def\t{\mathrm{t}}
\global\long\def\RR{\mathbb{R}}
\global\long\def\d{\,\mathrm{d}}
\global\long\def\U{\mathrm{U}}
\global\long\def\thb{\boldsymbol{\theta}}
\global\long\def\I{\mathrm{I}}
\global\long\def\II{\mathrm{II}}
\global\long\def\ein{\mathbf{1}}
\global\long\def\pv{p\text{-value}}
\global\long\def\MLE{\mathrm{MLE}}
\global\long\def\indep{\perp\!\!\!\perp}
\global\long\def\xib{\boldsymbol{\xi}}
\global\long\def\Pscr{\mathscr{P}}
\global\long\def\m{\mathsf{m}}



\title{Вычисление интегралов методом Монте-Карло}


\date{Tue 21 Feb 2017}


\author{422 группа}

\maketitle
\tableofcontents{}


\section{Общая схема}

Пусть $\nu:\A\to\bar{\RR}$ --- конечный заряд на пространстве $(D,\A)$.
Найдем $J=\nu(D)$. Для этого введем $\xi:(\Omega,\F,\P)\to(D,\A)$
--- случайную величина с распределением $\Pcal=\Pcal_{\xi}$, причем
$\nu\prec\Pcal$. 

$$
\xymatrix{
    (D, \A) \ar[r]^{\nu}_{\Pcal_\xi} & (\overline{\RR}, \mathcal{B}) \\
    (\Omega, \F, \P) \ar[u]^\xi \ar[ru]_{m \circ \xi = \eta}
}
$$

Значит существует производная Радона-Никодима 
\[
m=\frac{\d\nu}{\d\Pcal}:D\to\bar{\RR},
\]
т.е. такая функция $m(x)$, что
\[
\int_{A}m\d\Pcal=\int_{A}\d\nu=\nu(A),\quad\forall A\in\A.
\]
$J$ может быть найдена как 
\[
J=\E m(\xi)=\int_{\Omega}m(\xi)\d\P=\int_{D}m(x)\d\Pcal=\int_{D}\frac{\d\nu}{\d\Pcal}\d\Pcal=\int_{D}\d\nu=\nu(D).
\]
 По ЗБЧ может быть найдена оценка $\hat{J}_{n}$: если $\xi_{1},\ldots,\xi_{n}$
i.i.d., то 
\[
\hat{J}_{n}=\frac{1}{n}\sum_{i=1}^{n}m(\xi_{i})\toP\E m(\xi)=J.
\]



\section{Вычисление интеграла}

Посчитаем интеграл 
\[
\nu(A)=\int_{A}f\d\lambda.
\]
Автоматически, $\d\nu=f\d\lambda$ и $\nu\prec\lambda$. Пусть также
$\d\Pcal=p\d\lambda$, то есть $\Pcal\prec\lambda$. Таким образом,
мера $\Pcal$ и заряд $\nu$ абсолютно непрерывны относительно третьей
меры $\lambda$. Тогда для существования $m=\d\nu/\d\Pcal$ необходимо
и достаточно, чтобы $\lambda\left\{ x:f(x)\neq0\,\&\,p(x)=0\right\} =0$,
при этом 
\[
m(x)=\frac{\d\nu}{\d\Pcal}=\frac{f\d\lambda}{p\d\lambda}=\frac{f(x)}{p(x)}.
\]
 Таким образом, 
\[
\hat{J}_{n}=\frac{1}{n}\sum_{i=1}^{n}\frac{f(\xi_{i})}{p(\xi_{i})}.
\]



\section{Вычисление дисперсии}

Так как 
\[
\D m(\xi)=\E m^{2}(\xi)-\left(\E m(\xi)\right)^{2}=\E m^{2}(\xi)-J^{2},
\]
 то
\[
s_{n}^{2}=\frac{1}{n}\sum_{i=1}^{n}m^{2}(\xi_{i})-\hat{J}_{n}^{2}.
\]



\section{Асимптотический доверительный интервал}

Построение доверительного интервала 
\[
\P\left(\left|\hat{J}_{n}-J\right|\leq\epsilon\right)=1-\gamma.
\]
 По ЦПТ, 
\[
\frac{\sum\xi_{i}-\sum\E\xi_{i}}{\sqrt{\sum_{i}\D\xi_{i}}}=\frac{\sum\xi_{i}-n\mu}{\sqrt{n\sigma^{2}}}=\sqrt{n}\frac{1/n\cdot\sum\xi_{i}-\mu}{\sigma}=\frac{\sqrt{n}}{\sigma}\left(\hat{J}_{n}-J\right)\tod\N(0,1),
\]
 откуда 
\[
\P\left(-x\leq\frac{\sqrt{n}}{\sigma}\left(\hat{J}_{n}-J\right)\leq x\right)\tod\Phi(x)-\Phi(-x)=1-\gamma\iff x=x_{\gamma},
\]
и 
\[
\P\left(-x_{\gamma}\frac{\sigma}{\sqrt{n}}\leq\hat{J}_{n}-J\leq x_{\gamma}\frac{\sigma}{\sqrt{n}}\right),
\]
откуда доверительный интервал имеет вид 
\[
J\in\left(\hat{J}_{n}-\frac{\sigma}{\sqrt{n}}x_{\gamma};\hat{J}_{n}+\frac{\sigma}{\sqrt{n}}x_{\gamma}\right).
\]
 Неизвестную $\sigma$ можно заменить на выборочный стандарт $s_{n}$.


\section{Трудоемкость}

Найдем такое $n_{0}:\forall n\geq n_{0}$ $\P\left(\left|\hat{J}_{n}-J\right|\leq\epsilon\right)=1-\gamma$:
\[
x_{\gamma}\frac{\sigma}{\sqrt{n_{0}}}=\epsilon\implies n_{0}=\frac{\sigma^{2}}{\epsilon^{2}}x_{\gamma}^{2}.
\]


Итого, скорость сходимости $\mathcal{O}(1/\sqrt{n})$. Для реального
применения непригодны, потому что содержат неизвестную $\sigma$.
ЦПТ выдерживает подстановку состоятельную оценку дисперсии.


\section{Доверительный интервал через теорему Донскер}

... Объединение доверительных интервалов не является доверительным
интервалом для траектории ...

\[
\hat{J}_{n}=\frac{(n-1)\hat{J}_{n-1}+m(\xi_{n})}{n}
\]


Для таких вещей есть аналог ЦПТ: Пусть есть $x_{1},\ldots x_{n}$
i.i.d с $\mu=\E x_{i},\sigma^{2}=\D x_{i}$. Переведем $\hat{J}_{n}=\frac{1}{n}\sum_{i=1}^{n}x_{i}$
в интервал {[}0,1{]} :
\[
\delta_{n}(t)=\frac{1}{\left\lfloor nt\right\rfloor }\sum_{i=1}^{\left\lfloor nt\right\rfloor }x_{i},\quad t\in[0,1].
\]
Справедлива теорема Донскер: 
\[
\frac{1}{\sqrt{n}}\sum_{i=1}^{\left\lfloor nt\right\rfloor }\left(\frac{x_{i}-\mu}{\sigma}\right)\xrightarrow[n\to\infty]{\sim}W(t),
\]
 где $W(t)$ --- Броуновский мост. Как и ЦПТ, выдерживает замену $\sigma$
на $\tilde{s}$ . $\hat{J}_{n}=\delta_{n}(1)$, $x_{i}=m(\xi_{i})$
. 
\[
\frac{1}{\sqrt{n}\tilde{s}}\sum_{i=1}^{\left\lfloor nt\right\rfloor }\left(x_{i}-\delta(1)\right)-\underbrace{J}_{\mu}+J=\frac{1}{\sqrt{n}\tilde{s}}\sum_{i=1}^{\left\lfloor nt\right\rfloor }\left(x_{i}-J\right)-\frac{\left\lfloor nt\right\rfloor }{\sqrt{n}\tilde{s}}\left(\hat{J}_{n}-J\right)\xrightarrow[n\to\infty]{}W(t)+tU,\quad U\sim\N(0,1)
\]
(«броуновский мост со сносом»). Аналог доверительных интервалов для
процессов: 
\[
\left(\delta_{n}(1)-\frac{\tilde{s}\sqrt{n}}{\left\lfloor nt\right\rfloor }U^{**}(t),\delta_{n}(1)+\frac{\tilde{s}\sqrt{n}}{\left\lfloor nt\right\rfloor }U^{**}(t)\right),
\]
для 95\%-го доверительного интервала можно взять 
\[
U^{**}(t)=a+b\sqrt{t},\quad a=0.1,b=3.15.
\]

\end{document}
