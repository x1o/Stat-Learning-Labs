%% LyX 2.1.4 created this file.  For more info, see http://www.lyx.org/.
%% Do not edit unless you really know what you are doing.
\documentclass[russian,abstract=yes]{scrartcl}
\usepackage[T2A,T1]{fontenc}
\usepackage[utf8]{inputenc}
\usepackage[a4paper]{geometry}
\geometry{verbose,tmargin=2cm,bmargin=2.5cm,lmargin=2cm,rmargin=2cm}
\usepackage{color}
\usepackage{babel}
\usepackage{latexsym}
\usepackage{mathtools}
\usepackage{amsmath}
\usepackage{amsthm}
\usepackage{amssymb}
\usepackage{esint}
\usepackage[unicode=true,pdfusetitle,
 bookmarks=true,bookmarksnumbered=false,bookmarksopen=true,bookmarksopenlevel=1,
 breaklinks=false,pdfborder={0 0 1},backref=false,colorlinks=true]
 {hyperref}

\makeatletter

%%%%%%%%%%%%%%%%%%%%%%%%%%%%%% LyX specific LaTeX commands.
\DeclareRobustCommand{\cyrtext}{%
  \fontencoding{T2A}\selectfont\def\encodingdefault{T2A}}
\DeclareRobustCommand{\textcyr}[1]{\leavevmode{\cyrtext #1}}
\AtBeginDocument{\DeclareFontEncoding{T2A}{}{}}


%%%%%%%%%%%%%%%%%%%%%%%%%%%%%% Textclass specific LaTeX commands.
 \theoremstyle{definition}
 \newtheorem*{defn*}{\protect\definitionname}
  \theoremstyle{remark}
  \newtheorem*{rem*}{\protect\remarkname}
  \theoremstyle{definition}
  \newtheorem*{example*}{\protect\examplename}
  \theoremstyle{plain}
  \newtheorem*{thm*}{\protect\theoremname}
  \theoremstyle{plain}
  \newtheorem*{cor*}{\protect\corollaryname}
  \theoremstyle{remark}
  \newtheorem*{claim*}{\protect\claimname}

\@ifundefined{date}{}{\date{}}
%%%%%%%%%%%%%%%%%%%%%%%%%%%%%% User specified LaTeX commands.
%\usepackage{nicefrac}
%\usepackage{colortbl}
%\usepackage[noend]{algpseudocode}
%\usepackage[all]{xy}
\usepackage{mathrsfs}

%\usepackage[columns=1,itemlayout=singlepar,totoc=true]{idxlayout}

%\@addtoreset{chapter}{part}
\DeclareMathOperator{\Int}{Int}
\DeclareMathOperator{\rk}{rk}
\DeclareMathOperator*{\argmax}{argmax}
\DeclareMathOperator{\dist}{dist}
\DeclareMathOperator{\qnt}{qnt}
\DeclareMathOperator*{\rt}{root}

\newcommand{\bigperp}{%
  \mathop{\mathpalette\bigp@rp\relax}%
  \displaylimits
}

\newcommand{\bigp@rp}[2]{%
  \vcenter{
    \m@th\hbox{\scalebox{\ifx#1\displaystyle2.1\else1.5\fi}{$#1\perp$}}
  }%
}

\newcommand{\bignparallel}{%
  \mathop{\mathpalette\bignp@rp\relax}%
  \displaylimits
}

\newcommand{\bignp@rp}[2]{%
  \vcenter{
    \m@th\hbox{\scalebox{\ifx#1\displaystyle2.1\else1.5\fi}{$#1\nparallel$}}
  }%
}

\addto\captionsrussian{%
  \renewcommand{\abstractname}{TL;DR}%
}

\AtBeginDocument{
  \def\labelitemii{\(\Diamond\)}
  \def\labelitemiii{\(\Box\)}
}

\makeatother

  \providecommand{\claimname}{Утверждение}
  \providecommand{\corollaryname}{Следствие}
  \providecommand{\definitionname}{Определение}
  \providecommand{\examplename}{Пример}
  \providecommand{\remarkname}{Замечание}
  \providecommand{\theoremname}{Теорема}

\begin{document}
\global\long\def\N{\mathrm{N}}
\global\long\def\P{\mathsf{P}}
\global\long\def\E{\mathsf{E}}
\global\long\def\D{\mathsf{D}}
\global\long\def\O{\Omega}
\global\long\def\F{\mathcal{F}}
\global\long\def\K{\mathsf{K}}
\global\long\def\A{\mathscr{A}}
\global\long\def\Pcal{\mathcal{P}}
\global\long\def\th{\theta}
\global\long\def\toas{\xrightarrow{{\rm a.s.}}}
\global\long\def\toP{\xrightarrow{\P}}
\global\long\def\tod{\xrightarrow{\mathrm{d}}}
\global\long\def\iid{\mathrm{i.i.d.}}
\global\long\def\T{\mathrm{T}}
\global\long\def\L{\mathcal{L}}
\global\long\def\dd#1#2{\frac{\mathrm{d}#1}{\mathrm{d}#2}}
\global\long\def\a{\alpha}
\global\long\def\b{\beta}
\global\long\def\t{\mathrm{t}}
\global\long\def\RR{\mathbb{R}}
\global\long\def\d{\,\mathrm{d}}
\global\long\def\U{\mathrm{U}}
\global\long\def\thb{\boldsymbol{\theta}}
\global\long\def\I{\mathrm{I}}
\global\long\def\II{\mathrm{II}}
\global\long\def\ein{\mathbf{1}}
\global\long\def\pv{p\text{-value}}
\global\long\def\MLE{\mathrm{MLE}}
\global\long\def\indep{\perp\!\!\!\perp}
\global\long\def\xib{\boldsymbol{\xi}}
\global\long\def\Pscr{\mathscr{P}}
\global\long\def\m{\mathsf{m}}
\global\long\def\X{\mathfrak{X}}



\title{$M$-оценки и ОМП}


\date{Tue 28 Feb 2017}


\author{422 группа}

\maketitle
\tableofcontents{}


\section{$M$- и $Z$-оценки}

По выборке $\mathbf{x}=\left(x_{1},\ldots,x_{n}\right),$ оценим параметр
$\th\in\Theta$. Введем функцию-критерий 
\[
m_{\th}:\X\to\bar{\RR},
\]
показывающую насколько наблюдения соответствуют параметру и отображение
\[
\th\mapsto M_{n}(\th)=\frac{1}{n}\sum_{i=1}^{n}m_{\th}(x_{i}),
\]
сопоставляющее каждому параметру его <<подходящесть>>.
\begin{defn*}[$M$-оценка\footnote{<<Maximum>>}]
По принципу <<наилучшей хорошести>>, оценка есть 
\[
\hat{\th}_{n}=\argmax_{\th}M_{n}(\th).
\]
\end{defn*}
\begin{rem*}
Чтобы $\hat{\th}_{n}$ было случайной величиной, требуем, чтобы $\hat{\th}_{n}$
было измеримым, а для этого, чтобы $\Theta$ --- польским (полным,
измеримым, метрическим).
\end{rem*}

\begin{rem*}
Если взять $m=\log p_{\th}$, то $M$-оценка есть ОМП.\end{rem*}
\begin{defn*}[Z-оценка\footnote{<<Zero>>}]
Рассмотрим другие критерии $\psi_{\th}(x)$ и отображение 
\[
\th\mapsto\Psi_{n}(\th)=\frac{1}{n}\sum_{i=1}^{n}\psi_{\th}(x_{i}).
\]
В качестве оценки найдем 
\[
\hat{\th}_{n}=\rt_{\th}\Psi_{n}(\th).
\]
\end{defn*}
\begin{rem*}
$M$-оценка может быть сведена к этой оценке $\psi_{\th}=\partial m_{\th}/\partial\th$.
\end{rem*}

\begin{rem*}
$\psi_{\th}$ должна быть, конечно, дифференцируема.\end{rem*}
\begin{example*}
Пусть $x_{1},\ldots,x_{n}$ $\iid$ на $\U[0,\th)$, то $p_{\th}=\th^{-1}\ein_{[0,\th)}$
и 
\[
\th\mapsto\sum_{i=1}^{n}(\log\ein_{[0,\th)}(x_{i})-\log(\th)).
\]
Максимум z-оценки достигается на 
\[
\hat{\th}_{n}=\max x_{i}.
\]
 ОМП оценку не найти, потому что не продифференцировать индикатор.
\end{example*}

\section{Состоятельность}

Пусть есть 
\[
M_{n}(\th)=\frac{1}{n}\sum_{i=1}^{n}m_{\th}(x_{i})\xrightarrow[n\to\infty]{\P}\E m_{\th}=M(\th)=\int m_{\th}\d\Pcal_{x}.
\]
 Навесим $\argmax$ на обе стороны: 
\[
M_{n}(\th)\rightsquigarrow\hat{\th}_{n}=\argmax_{\th}M_{n}(\th).
\]
Если бы $\argmax$ было непрерывным, то $\hat{\th}_{n}\xrightarrow[n\to\infty]{\P}\th^{*}=\argmax M(\th).$
Нужно проверить следующие вещи:
\begin{enumerate}
\item Сходится ли $\hat{\th}_{n}$ хоть к чему-нибудь.
\item Правда ли, что $\th^{*}=\th_{0}$.
\end{enumerate}
$\argmax$ в принципе не является непрерывным. До тех пор, пока находимся
в области притяжения параметра, всё хорошо, но как только из нее выходим,
можем резко перескочить на другой экстремум.


\subsection{Сходимость к чему-нибудь}
\begin{thm*}[Вальд]
Рассматриваем $M$-оценку. Пусть
\begin{enumerate}
\item $m_{\th}$ полунепрерывна сверху по $\th$ для почти всех $x$: 
\[
\lim_{\th\to\th_{0}}\sup\leq m_{\th_{0}},\quad\text{для почти всех }x.
\]
 (если экстремум в $\th_{0}$, то он хорошо выражен).
\item $\argmax$ должно быть случайной величиной. Для этого отображение
\[
\forall B_{\delta}\subset\Theta\ x\mapsto\sup_{\th\in B_{\delta}}m_{\th}(x)
\]
 должно быть

\begin{enumerate}
\item измеримо
\item ограничнено: 
\[
\int\sup_{B_{\delta}}m_{\th}(x)\d\P<\infty.
\]

\end{enumerate}
\end{enumerate}
\end{thm*}
\begin{rem*}
Пусть $\Theta^{*}=\left\{ \th^{*}\in\Theta:M(\th^{*})=\sup_{\th}M(\th)\right\} .$
Еще ослабим условие: максимум будет достигаться асимптотически. Пусть
интересует 
\[
\hat{\th}_{n}=\argmax(M_{n}(\th)-o(1))
\]
 (не доходим до максимума --- с точки зрения приложений ок, потому
что ищем оценки численно). Т.е. интересуемся 
\[
M_{n}(\hat{\th}_{n})\geq c\sup_{\th}M_{n}(\th),\quad c\xrightarrow[n\to\infty]{}1,\ 0\leq c<1.
\]
 Тогда для любой такой последовательности оценок, $\forall\epsilon>0\ \forall K\subset\Theta$
($K$ --- компакт), верно, что 
\[
\P\left(\dist\left(\hat{\th}_{n},\th^{*}\right)\geq\epsilon,\hat{\th}_{n}\in K\right)\xrightarrow[n\to\infty]{}0.
\]
Иными словами, это условие на то, как оценка не сходится --- либо
последовательность сходится, либо выходит за компакт.\end{rem*}
\begin{cor*}
Поэтому для $\RR^{n}$ всё либо сходится по вероятности, либо уходит
на бесконечности --- \emph{не может колебаться} на манер $\left(-1\right)^{n}$.
Это очень здорово.
\end{cor*}

\subsection{Равенство $\theta^{*}=\theta_{0}$}

Говорили, что ОМП получаются, когда 
\[
M_{n}(\th)=\frac{1}{n}\sum_{i=1}^{n}\log p_{\th}(x_{i}).
\]
Совершим трюк: добавим константу, так что 
\[
M_{n}(\th)=\frac{1}{n}\sum_{i=1}^{n}\log\frac{p_{\th}(x_{i})}{p_{\th_{0}}(x_{i})}.
\]
 На максимум не влияет, конечно. Тогда 
\[
\frac{1}{n}\sum_{i=1}^{n}\ln\frac{p_{\thb}}{p_{\thb_{0}}}(x_{i})\toP\int_{\RR}\ln\left(\frac{p_{\thb}(x)}{p_{\thb_{0}}(x)}\right)p_{\thb_{0}}(x)\d x=\E\ln\frac{p_{\thb}}{p_{\thb_{0}}}\leq\ln\E\frac{p_{\thb}}{p_{\thb_{0}}}=\ln\int_{\RR}\frac{p_{\thb}}{p_{\thb_{0}}}(x)p_{\thb_{0}}(x)\d x=\ln1=0
\]
 так что 
\[
M(\th)\leq0.
\]
 Интересуемся $\th:M(\th)=0$. Это так, когда $p_{\th}=p_{\th_{0}}$
для почти всех $x$. В предположении свойства \emph{идентифицируемости}
задачи ($\th_{1}\neq\th_{2}\implies\Pcal_{\th_{1}}\neq\Pcal_{\th_{2}}$),
получаем $\th=\th_{0}.$

Таким образом, нужно иметь адекватную модель (выполнялось бы условие
идентифицируемости). Для выполнимость теоремы Вальда достаточно компактности
и $\sqrt{\log p}$\ldots{} Компактность на практике выполняется ---
можно взять достаточно большой отрезок всегда.

Так что состоятельность выполняется.


\section{Скорость сходимости}

При выполнении некоторых условий регулярности (простые почти никогда
не выполняются, сложные сложно проверить) ОМП асимптотически нормальны,
т.е. 
\[
\sqrt{n}(\hat{\th}_{n}-\th_{0})\tod\N(0,I^{-1}(\th_{0})).
\]


Если условия регулярности не выполняются, может быть все что угодно.
Обычно, если $\Theta=\RR^{n}$, то оценки сходятся оч быстро. Иначе
медленно.


\section{Доверительные границы}


\subsection{Аппроксимация Вальда}

Исходя из нормальной аппроксимации, доверительные границы можно 
\[
\left(\hat{\th}_{n}-\th_{0}\right)^{\T}\Sigma^{-1}\left(\hat{\th}_{n}-\th\right)\tod\chi_{d}^{2}.
\]
 Проблемы:
\begin{itemize}
\item Границы лишь асимптотические.
\item Нужно знать $\Sigma$, зависящую от оцениваемого параметра. Поэтому
нужно еще и оценивать $\Sigma^{-1}$.
\end{itemize}
Так можно делать в низких размерностях и при больших выборках.


\subsection{Метод профилей правдоподобия}

Пусть есть $\thb,\dim\thb=k$. НУО, выделим $\thb=(\thb_{1},\thb_{2}),\dim\thb_{1}=k_{1}.$
Рассмотрим логарифм функции правдоподобия как функцию двух аргументов
\[
\ell(\thb)=\ell(\thb_{1},\thb_{2}).
\]
 Построим \emph{профиль} 
\[
r(\thb_{2})=\ell(\hat{\thb}_{1}(\thb_{2}),\thb_{2}),
\]
 где $\hat{\thb}_{1}(\thb_{2})$ --- MLE оценка $\thb_{1}$ с данным
$\thb_{2}$: $\hat{\thb}_{1}(\thb_{2})=\argmax_{\thb_{1}}\ell(\thb_{1},\thb_{2})$.
\begin{rem*}
Если $\thb_{2}=\hat{\left(\thb_{2}\right)}_{\MLE}$, то $\hat{\thb}_{1}\hat{\left(\thb_{2}\right)}_{\MLE}=\hat{\left(\thb_{1}\right)}_{\MLE}.$\end{rem*}
\begin{claim*}
Пусть истинный параметр $\thb_{0}=(\thb_{10},\thb_{20})$. Тогда 
\[
z^{2}:=-2\left(r(\thb_{10})-\ell(\hat{\thb}_{n})\right)\tod\chi^{2}(k_{1}).
\]
\end{claim*}
\begin{proof}[Идея доказательства]
Знаем, что $\hat{\thb}_{n}$ имеет нормальное распределение, $\ell(\hat{\thb}_{n})\sim\chi^{2}.$
Тогда в пределе $\ell$ должна выглядеть как сумма квадратов --- можно
разложить по Тейлору, будет что-то вроде 
\[
\dots+\frac{1}{2}\left(\thb_{0}-\hat{\thb}_{n}\right)^{2}+\dots
\]
 Условия регулярности нужны чтобы убить все члены старше второго.
\end{proof}
Нужно уметь обращать $r$, чтобы 
\[
\thb_{10}:\qnt_{\chi^{2}}(\gamma_{1})\leq-2\left(\dots\right)\leq\qnt_{\chi^{2}}(\gamma_{2}).
\]
Так что заменили задачу оценивания $\Sigma^{-1}$ на задачу обращения
детерминированной функции $r$. В качестве бонуса --- профиль должен
быть похож на квадратичную функцию, тогда доверительный интервал по
квантилям будет правильно выбран (частный случай <<локальной асимптотической
нормальности>>).

На глаз можно посмотреть график $\left|z\right|$. Когда всё хорошо,
он будет выглядеть как модуль. 
\begin{example*}
Пусть $\xi\sim\N(\mu_{0},\sigma_{0}^{2})$, $\thb=(\mu,\sigma^{2})$.
Пусть $\mu_{0}$ известна; оценим по выборке $\sigma^{2}$. Т.к. $\chi^{2}(1)=(\N(0,1))^{2}$,
то 
\[
z^{2}=-2(\underbrace{r(\mu_{0})}_{\max_{\sigma^{2}}\ell(\mu_{0},\sigma^{2})}-\ell(\mu_{0},\widehat{\sigma^{2}}))\tod\left(\N(0,1)\right)^{2}.
\]
 Значит $z\tod\N(0,1)$. Ясно, что в $\widehat{\sigma^{2}}=\widehat{\sigma^{2}}_{\MLE},$
$\left|z\right|=0$. На остальных значениях это модуль-галочка, причем
по значениям $\left|z\right|$ --- квантилям $\N(0,1)$ --- можно
построить доверительный интервал для $\widehat{\sigma^{2}}$.\end{example*}
\begin{rem*}
В R --- пакет \texttt{bbmle}, функция \texttt{mle2}. На выходе ---
объект типа \texttt{mle}. Это оценки + \ldots{}. \texttt{summary(mle)}.
Построить профиль можно при помощи функции \texttt{p <- profile(m)}.
Вызывает \texttt{optim} и т.п. Профили можно рисовать:\texttt{ plot(p)}.
Подсчет доверительных интервалов: \texttt{confint(p)}.\end{rem*}

\end{document}
